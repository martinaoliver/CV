%%%%%%%%%%%%%%%%%
% This is an sample CV template created using altacv.cls
% (v1.7, 9 August 2023) written by LianTze Lim (liantze@gmail.com). Compiles with pdfLaTeX, XeLaTeX and LuaLaTeX.
%
%% It may be distributed and/or modified under the
%% conditions of the LaTeX Project Public License, either version 1.3
%% of this license or (at your option) any later version.
%% The latest version of this license is in
%%    http://www.latex-project.org/lppl.txt
%% and version 1.3 or later is part of all distributions of LaTeX
%% version 2003/12/01 or later.
%%%%%%%%%%%%%%%%

%% Use the "normalphoto" option if you want a normal photo instead of cropped to a circle
% \documentclass[10pt,a4paper,normalphoto]{altacv}

\documentclass[10pt,a4paper,ragged2e,withhyper]{altacv}
%% AltaCV uses the fontawesome5 and packages.
%% See http://texdoc.net/pkg/fontawesome5 for full list of symbols.

% Change the page layout if you need to
\geometry{left=1.25cm,right=1.25cm,top=1.5cm,bottom=1.5cm,columnsep=1.2cm}

\geometry{left=1cm,right=1cm,top=1cm,bottom=1cm,columnsep=1.1cm}

% The paracol package lets you typeset columns of text in parallel
\usepackage{paracol}

% Change the font if you want to, depending on whether
% you're using pdflatex or xelatex/lualatex
% WHEN COMPILING WITH XELATEX PLEASE USE
% xelatex -shell-escape -output-driver="xdvipdfmx -z 0" sample.tex
\ifxetexorluatex
  % If using xelatex or lualatex:
  \setmainfont{Roboto Slab}
  \setsansfont{Lato}
  \renewcommand{\familydefault}{\sfdefault}
\else
  % If using pdflatex:
  \usepackage[rm]{roboto}
  \usepackage[defaultsans]{lato}
  % \usepackage{sourcesanspro}
  \renewcommand{\familydefault}{\sfdefault}
\fi

% Change the colours if you want to
\definecolor{SlateGrey}{HTML}{2E2E2E}
\definecolor{LightGrey}{HTML}{666666}
\definecolor{DarkPastelRed}{HTML}{450808}
\definecolor{PastelRed}{HTML}{8F0D0D}
\definecolor{GoldenEarth}{HTML}{E7D192}
\definecolor{Green}{HTML}{0d8f44}

\definecolor{DarkBlue}{HTML}{073949}
\definecolor{Turquoise}{HTML}{0d848f}
\colorlet{name}{black}
\colorlet{tagline}{Turquoise}
\colorlet{heading}{DarkBlue}
\colorlet{headingrule}{DarkBlue}
\colorlet{subheading}{Turquoise}
\colorlet{accent}{Turquoise}
\colorlet{emphasis}{SlateGrey}
\colorlet{body}{LightGrey}

% Change some fonts, if necessary
\renewcommand{\namefont}{\LARGE\rmfamily\bfseries}
\renewcommand{\personalinfofont}{\footnotesize}
\renewcommand{\personalinfotwofont}{\footnotesize}
\renewcommand{\cvsectionfont}{\Large\rmfamily\bfseries}
\renewcommand{\cvsubsectionfont}{\large\bfseries}


% Change the bullets for itemize and rating marker
% for \cvskill if you want to
\renewcommand{\cvItemMarker}{{\small\textbullet}}
\renewcommand{\cvRatingMarker}{\faCircle}
% ...and the markers for the date/location for \cvevent
% \renewcommand{\cvDateMarker}{\faCalendar*[regular]}
% \renewcommand{\cvLocationMarker}{\faMapMarker*}


% If your CV/résumé is in a language other than English,
% then you probably want to change these so that when you
% copy-paste from the PDF or run pdftotext, the location
% and date marker icons for \cvevent will paste as correct
% translations. For example Spanish:
% \renewcommand{\locationname}{Ubicación}
% \renewcommand{\datename}{Fecha}


%% Use (and optionally edit if necessary) this .tex if you
%% want to use an author-year reference style like APA(6)
%% for your publication list
% % When using APA6 if you need more author names to be listed
% because you're e.g. the 12th author, add apamaxprtauth=12
\usepackage[backend=biber,style=apa6,sorting=ydnt]{biblatex}
\defbibheading{pubtype}{\cvsubsection{#1}}
\renewcommand{\bibsetup}{\vspace*{-\baselineskip}}
\AtEveryBibitem{%
  \makebox[\bibhang][l]{\itemmarker}%
  \iffieldundef{doi}{}{\clearfield{url}}%
}
\setlength{\bibitemsep}{0.25\baselineskip}
\setlength{\bibhang}{1.25em}


%% Use (and optionally edit if necessary) this .tex if you
%% want an originally numerical reference style like IEEE
%% for your publication list
\usepackage[backend=biber,style=ieee,sorting=ydnt,defernumbers=true]{biblatex}
%% For removing numbering entirely when using a numeric style
\setlength{\bibhang}{1.25em}
\DeclareFieldFormat{labelnumberwidth}{\makebox[\bibhang][l]{\itemmarker}}
\setlength{\biblabelsep}{0pt}
\defbibheading{pubtype}{\cvsubsection{#1}}
\renewcommand{\bibsetup}{\vspace*{-\baselineskip}}
\AtEveryBibitem{%
  \iffieldundef{doi}{}{\clearfield{url}}%
}


%% sample.bib contains your publications
\addbibresource{sample.bib}

\begin{document}
\name{Martina Oliver Huidobro, PhD}

\tagline{Computational Biologist}

%% You can add multiple photos on the left or right
% \photoR{2.8cm}{Globe_High}
% \photoL{2.5cm}{Yacht_High,Suitcase_High}

\personalinfo{%
  % Not all of these are required!
  \email{martinaoliver1@gmail.com}\phone{+447883738907}\linkedin{martinaoliver}}

\personalinfotwo{%
  % Not all of these are required!
  \github{martinaoliver}\nationality{Spanish, Argentinian}\location{London, UK}}

  % \mailaddress{Åddrésş, Street, 00000 Cóuntry}
  % \homepage{www.homepage.com}
  % \twitter{@twitterhandle}
  
  % \orcid{0000-0000-0000-0000}
  %% You can add your own arbitrary detail with
  %% \printinfo{symbol}{detail}[optional hyperlink prefix]
  % \printinfo{\faPaw}{Hey ho!}[https://example.com/]

  %% Or you can declare your own field with
  %% \NewInfoFiled{fieldname}{symbol}[optional hyperlink prefix] and use it:
  % \NewInfoField{gitlab}{\faGitlab}[https://gitlab.com/]
  % \gitlab{your_id}
  %%
  %% For services and platforms like Mastodon where there isn't a
  %% straightforward relation between the user ID/nickname and the hyperlink,
  %% you can use \printinfo directly e.g.
  % \printinfo{\faMastodon}{@username@instace}[https://instance.url/@username]
  %% But if you absolutely want to create new dedicated info fields for
  %% such platforms, then use \NewInfoField* with a star:
  % \NewInfoField*{mastodon}{\faMastodon}
  %% then you can use \mastodon, with TWO arguments where the 2nd argument is
  %% the full hyperlink.
  % \mastodon{@username@instance}{https://instance.url/@username}


\makecvheader
%% Depending on your tastes, you may want to make fonts of itemize environments slightly smaller
% \AtBeginEnvironment{itemize}{\small}

%% Set the left/right column width ratio to 6:4.
\columnratio{0.6}

% Start a 2-column paracol. Both the left and right columns will automatically
% break across pages if things get too long.
\begin{paracol}{2}
\cvsection{Education}

\educationevent{PhD in Mathematical and Computational Biology}{Imperial College,  Centre for Integrative Systems and
Bioinformatics}{Oct 2020 -- April 2024}{CDT Scholarship, PhD}
\href{https://doi.org/10.25560/111289}{\textbf{Thesis: "Data-driven modelling of robust Turing patterns in synthetic biofilms \faLink"}}
\begin{itemize}
\item Developed and validated models of gene expression using mathematical tools such as PDE numerical solvers and stability analysis. 

\item Used machine learning techniques: Bayesian inference and regression methods for parameter fitting; neural networks for image clustering; PCA and t-SNE for understanding the parameter space; and MCMC for system optimisation.

\item Generated and analysed large biological datasets using High Performance Computing clusters and multi-threading, and built SQL databases to store and query results.

\item \href{https://papers.ssrn.com/sol3/papers.cfm?abstract_id=4733248}{Collaborated with experimental biologists to validate models and guide experimental design for the production of novel biomaterials.}

\item Speaker at 4 international conferences in mathematical biology and received 2 best poster awards (over 200 candidates). 

\item Published in 3 high-impact scientific journals: \faLink \href{https://papers.ssrn.com/sol3/papers.cfm?abstract_id=4733248}{[1] Cell Systems}  
\href{https://doi.org/10.1111/1751-7915.13979}{[2] Journal of microbial biotech.} \href{https://www.biorxiv.org/content/10.1101/2024.09.09.611947v1}  {[3] PLOS Computational Biology (in review)} 
\href{https://doi.org/10.25560/111289}{[4] Spiral PhD Thesis} and \href{https://www.sciencedirect.com/science/article/pii/S2405471222004367}{peer reviewed other publications.} 


\end{itemize}
\vspace{-0.5\baselineskip} % Adjust space as needed

\divider

\educationevent{MRes in Systems and Synthetic Biology}{Imperial College London, CDT BioDesign Engineering}{Oct 2019  -- Sept 2020}{CDT Scholarship, Distinction 75\%}
\textbf{Relevant courses:} Maths, Informatics and Statistics for Computational Biology (SysMIC); Introduction to machine learning (Yandex School of Data Science); Differential equations (Boston University). 

\vspace{0.2cm}

\textbf{Thesis:} "High throughput study and optimization of synthetic gene networks for pattern formation in tissue engineering."

\divider


\educationevent{BSc in Biochemistry and Bioinformatics}{Imperial College London, Life Sciences Department}{Oct 2016 -- July 2019}{First Class 71\%} 

\textbf{Core modules:} Bioinformatics, Integrative Systems Biology, Metabolic Network Engineering.
\vspace{0.2cm}

\textbf{Thesis:} "Molecular dynamics simulations of $\alpha$-synuclein in Parkinson's disease." Predicted protein structure and dynamics using GROMACS software and pyMOL visualization.  

\vspace{0.2cm}
\textbf{Lit Review:} "Multiscale modelling of Tumour-Immune interactions."


\vspace{-0.5\baselineskip} % Adjust space as needed
\divider

\educationevent{European Baccalaureate -- High school degree}{European School of Brussels I}{2008-2016}{Highest grade of cohort 91.6\%}
 \textbf{Core modules}: Advanced Mathematics, Physics, Biology, Chemistry.
 \medskip

\cvsection{Additional Courses}

\begin{itemize}
    \item \textbf{Genomic Data Science Specialization} - Johns Hopkins University
        \item \textbf{Advanced Learning algorithms} - Stanford University
        \item \textbf{Supervised machine learning} - Stanford University
\end{itemize}



\switchcolumn

\cvsection{Work Experience}

\cvevent{Events Lead, Activator team}{Nucleate}{March 2023 -- February 2024}{London}
\begin{itemize}
\item Organised events e.g. competition with UK university spinouts pitching for a £2M prize.
\end{itemize}
\vspace{-1.5\baselineskip} % Adjust space as needed
\divider

\cvevent{Teaching and project supervision}{Imperial College London}{Oct 2020 -- March 2024}{London}
\begin{itemize}
\item Supervised 20 students carrying out their BSc and MSc research projects in mathematical and computational biology.
\item Taught courses such as Programming for systems biology, Bioinformatics, Integrative systems biology, Maths.
\end{itemize}
\vspace{-0.5\baselineskip} % Adjust space as needed

\divider

\cvevent{Co-founder, start-up for early-stage Alzheimer’s diagnostics}{miCHIP}{Jan 2018--March 2019, April 2022}{London}
\begin{itemize}
\item As an undergrad, secured £6k funding and lab space to develop proof of concept.
    \item Obtained proof of concept for an early-stage diagnostics method
for Alzheimer’s, gaining experimental and computational experience with RNA synthesis and RNA structure prediction. 
\item Finalists in start-up competitions: FONS-MAD, WE Innovate, SynBioUK Catalyse. News highlights: \faLink \href{https://www.imperial.ac.uk/news/187629/four-student-ideas-that-could-change/}{[1]} \href{https://www.imperial.ac.uk/news/190372/five-women-led-startups-building-better-future/}{[2]}
\end{itemize}

\medskip
\cvsection{Additional Projects}
\begin{itemize}
    \item 2nd place at the EF Bio x AI Hackathon: developed multimodal data and machine learning approach for biodiversity mapping.
    
    \item SOTA Hackathon: Deep learning model to create novel carbon-capturing enzymes.
    \item Peer reviewed scientific documents for the Convention on Biological Diversity (CBD).
\end{itemize}
 \medskip

\cvsection{Skills}
\printinfo{\faFile*[regular]}{\textbf{Hard Skills}}
\vspace{0.2cm}

\cvtag{Differential equations}
\cvtag{Machine learning}
\cvtag{Python}
\cvtag{TensorFlow}
\cvtag{scikit-learn}
\cvtag{Biopython}
\cvtag{SQL}
\cvtag{HPC}
\cvtag{Bash}
\cvtag{Git}
\cvtag{GROMACS}
\vspace{0.1\baselineskip} % Adjust space as needed

\divider


\printinfo{\faFile*[regular]}{\textbf{Languages}}
\vspace{0.2cm}

\cvtag{Spanish (Fluent)}
\cvtag{English (Fluent)}\\
\cvtag{French (Fluent)}
\cvtag{Portuguese (Beginner)}






\end{paracol}


\end{document}


\cvsection{A Day of My Life}

% Adapted from @Jake's answer from http://tex.stackexchange.com/a/82729/226
% \wheelchart{outer radius}{inner radius}{
% comma-separated list of value/text width/color/detail}
\wheelchart{1.5cm}{0.5cm}{%
  6/8em/accent!30/{Sleep,\\beautiful sleep},
  3/8em/accent!40/Hopeful novelist by night,
  8/8em/accent!60/Daytime job,
  2/10em/accent/Sports and relaxation,
  5/6em/accent!20/Spending time with family
}

% use ONLY \newpage if you want to force a page break for
% ONLY the current column
\newpage
